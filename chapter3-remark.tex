\section{Introduction}
第三章讨论了非相对论量子力学中的\C \P \T 的问题,我的注记包括CPV一书和井樱,以及Weinberg书中的内容

\section{Ray}
Weinberg给出了量子力学中Ray的概念
Ray表示一组量子态的集合,它们相差一个相位,我们把只有一个相位差的态看作是物理上不可区分的。

\section{对称变换}
对称变换的意思是不改变实验结果的变换。如果观察者$O$看到的态可以表示为$R$,$R_1$,$R_2$,观察者$O^\prime$可以表示为
$R^\prime$,$R_1^\prime$,$R_2^\prime$,两个观察者必须能看到相同的概率
\begin{equation}
    P(R\rightarrow R_n) = P(R^\prime\rightarrow R_n^\prime)
\end{equation}

应该注意到的是,这个要求是一个对称变换的唯一要求,这也是对量子力学中可观测概念的另一个印证。
Wigner的对称表示定理给出,一个对称变化或者是线性幺正算符,或者是反幺正算符
在讨论到反幺正算符的时候,我们这样定义我们的态和变换:
\begin{equation}
    \overrightarrow A\ket{\psi} = A\ket{\psi}
\end{equation}
\begin{equation}
    \bra{\psi}A = \bra{\psi}\overleftarrow A = (A\ket{\psi})^\dagger
\end{equation}
用这种记号,我们将两个态的内积可以定义为
\begin{align}
    \braket{\psi A}{A\psi} &= \braket{\phi}{\psi} \\
    A(\xi\ket{\phi} + \eta\ket{\psi}) &= \xi A\ket{\phi} + \eta A\ket{\psi}
\end{align}
反幺正反线性算符可以定义为
\begin{align}
    \braket{\phi A}{A\phi} &= \braket{\phi}{\psi} \\
    A(\xi\ket{\phi} +\eta\ket{\psi}) &= \xi^* A\ket{\phi} + \eta^* A\ket{\psi}
\end{align}

我们用$T$来表示Ray之间的变换,$U = U(T)$表示态之间的变换。
考虑这样一个变化
$T$应该形成一个群。
\begin{equation}
    U(T_1)U(T_2)\ket{\phi_n} = e^{i\phi_n(T_1, T_2)}U(T_2T_1)\ket{\phi_n}
\end{equation}
考虑态
\begin{equation}
    \ket{\psi_{AB}} = \ket{\psi_A} + \ket{\psi_B}
\end{equation}
那么,
\begin{align}
    e^{i\phi_{AB}}U(T_1T_2)\ket{\psi_{AB}} &= U(T_1)U(T_2)(\ket{\psi_A} + \ket{\psi_B}) \\
&= U(T_1)U(T_2)(\ket{\psi_A} + \ket{\psi_B}) \\
&= e^{i\phi_A}U(T_2T_1)\ket{\psi_A} + e^{i\phi_B}U(T_2T_1)\ket{\psi_B} 
\end{align}
由此可得
\begin{equation}
e^{i\phi_{AB}}(\ket{\psi_A} + \ket{\psi_B})
= e^{i\phi_A}\ket{\psi_A} + e^{i\phi_B}\ket{\psi_B}
\end{equation}
因此,只有在$\phi_{AB} = \psi_A = \psi_B$时才能成立。



\section{对称表示定理(The Symmetry Representation Theorem)}
定理:任意对称变换可以用Hiblert空间中的一个线性幺正算符或反线性反幺正算符表示

\section{变换}
先尝试用主动与被动的观点阐述变换的问题
\subsection{被动变换}
在我看来,被动变换的表述是更加简单的
\subsubsection{经典标量场}
在$O$参考系下,场$\phi$可以用$\phi = \phi(x)$表述,我们将转换到参考系$\bar{O}$,为了与之后讨论的主动变换结论一致,我们希望参考系之间的变换为$O^\prime = \Lambda O$,其中$\Lambda$表示一个Lorentz变换,由于$Lorentz$变换形成一个群,所以我们取$\Lambda$或者$\Lambda^{-1}$并不是太有关系。在新的参考系中,场$\phi$应该表示为$\bar \phi(\bar x)$,事实上,$phi$和$\bar \phi$表示的是同一个点,如果在$x$处$\phi$可以渠道极致,那么对应地,$\bar \phi$也应该在$\bar x$处取到极值。即应该有$\phi(x) = \bar\phi(\bar x)$。带入$\bar x = \Lambda^{-1} x$,应该有
\begin{equation}
    \bar\phi(\bar x) = \phi(x) = \phi(\Lambda \bar x)
\end{equation}
可以记为
\begin{equation}
    U\phi(x) = \bar\phi(x) = \phi(\Lambda x)
\end{equation}

\subsection{主动变换}
\subsection{经典标量场}
让场做一个$\Lambda$的Lorentz变换,于是,原来在$x$处的值就变成了$\Lambda x$处的值。
即
\begin{equation}
    \bar phi(\Lambda) = \phi(x)
\end{equation}

在后面的章节里面,我们将仔细讨论用$\UC,\invUC$等符号表示态变换,而用$\C \P$等表示对算符的变换,应该注意到,$\UC,\invUC$等的意义是确定性的,而$\C,\P$等符号的意义取决于对应的算符,在非相对论量子力学中,我们关心的主要是在$\UC,\invUC$等对态变换的作用下态变换的形式,在第4章中,我们会对算符的变换做进一步的讨论。

\section{\C 变换}
电荷共轭变换在非相对论量子力学不容易给出,讨论电荷共轭变化,必须考虑电磁场

讨论含电磁场的拉式量
\begin{equation}
H = \frac{|\vec P|^2}{2m} - \frac e{2mc}[\vec A \cdot \vec P + \vec P \cdot \vec A] + \frac{e^2}{2mc^2}\vec A^2 + e\phi
\end{equation}
做下面的变换后,哈密顿算符是不变的:
\begin{equation}
    e\rightarrow-e, \vec A \rightarrow -\vec A, \phi\rightarrow -\phi
\end{equation}
同时,$\vec P$以及$\vec X$是不变的。
因此,我们应该将\C 定义为使得系统的$e,A,\phi$反号的变换。
显然有$\C^2 = 1$实际上,\C 变换的定义比较复杂,需要在有Dirac方程的时候做进一步的讨论。
\section{\P 变换}
在量子力学中讨论\P 变换,\P 变换将右手参考系变换为左手参考系。
将量子力学的态$\ket{\alpha}$转化为$\ket{\alpha}\rightarrow\P\ket{\alpha}$
位置算符的平均值应该有
\begin{equation}
    \bra{\alpha}\invUP\vec x \UP\ket{\alpha} = - \bra{\alpha}\vec x\ket{\alpha}
\end{equation}
因此应该有
\begin{equation}
    \UP x\invUP = -x
\end{equation}
\section{\T 变换}

